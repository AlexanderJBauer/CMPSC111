\documentclass[12pt]{article}

\usepackage{fancyhdr}
\usepackage{amsthm}
\usepackage{amsmath}
\usepackage{graphicx}
\usepackage{amssymb}
\usepackage{esint}
\usepackage{subfigure}
\usepackage{color}
\usepackage{moreverb}
\usepackage{wrapfig}
\usepackage{csquotes}
\usepackage{url}
\graphicspath{ {images/} }
\textwidth 17cm \topmargin -1cm \oddsidemargin 0cm \textheight 21.5cm
\pagestyle{empty} \pagestyle{fancyplain}
\lhead[\fancyplain{}{}]{\fancyplain{}{{\sc Alexander Bauer}}}
\chead[\fancyplain{}{}]{\fancyplain{}{{\sc Homework 03 Part 2}}}
\rhead[\fancyplain{}{}]{\fancyplain{}{{\sc Computer Science 111}}}

\newcommand{\etal}{\textit{et al. }}

\begin{document}
\centerline{\Large\textbf{Diffusion Simulation}}
\vspace{0cm}


\section{Introduction}\label{sec::Intro}
This is a simulation of advection. Advection is the movement of a substance, in which the properties of the substance are carried with it even though it is moving. Generally the advected substance is a fluid. The properties that are carried along are things like energy. The advection occurs due to a velocity vector field. A good way to imagine advection is to think of a drop of ink being carried off by a river. For this simulation the substance of the material being advected is unknown, and what is causing the velocity vector field is unkown as well. What we do know is this:

\begin{enumerate}

\item The domain of the simulation is in two spacial dimensions:
$$ \Omega=[\frac{-\pi}{2},\frac{\pi}{2}]^2 $$
\item The velocity vector field \textbf(U) is given by:
$$ \textbf{U} = (u,v)^T $$ where $$ u(x,y,t) = -cos(x)sin(y)cos(t) $$ 
$$ v(x,y,t)=sin(x)cos(y)cos(t) $$
\item We want to find the solution to $\rho=\rho(x,y,t)$ where $\rho$ satisfies the advection equation:
$$ \frac{\delta \rho}{\delta t} + \textbf{U}\cdot \bigtriangledown \rho = 0 $$ with boundary conditions:
$$ \begin{aligned}
 \rho(x=-\pi / 2) = 0 \\
 \rho(x=+\pi / 2) = 0 \\
 \rho(y=-\pi / 2) = 0 \\
 \rho(y=+\pi / 2) = 0 \\
 \end{aligned}
 $$
 
\end{enumerate}

In order to simulate the system given, I set up the initial condition given by these conditions:
$$ 
\begin{aligned}
 \rho(x,y,t=0) = 1 \texttt{ iff } (x,y)\in C, \\
 \rho(x,y,t=0) = 0 \texttt{ otherwise} \\
\end{aligned} 
$$
where \emph{C} is a circle with center (1,0) and radius .25.

Then I implement the upwind scheme for the advection equation in order to solve for $ \rho $ at $ t_{final}=\pi $. The simulation discretises the domain into 100 grid nodes in each spacial direction. To iterate through the simulation until $ t_{final} $ a time step of $ \Delta t=.2\Delta x $ is used. Where $ \Delta x = x_{max} - x_{min} / N $ where $ N = 100 $ and $ \Delta x = \Delta y $.

\section{Algorithms}
I implement an upwind scheme to solve the equation numerically. The upwind scheme I used for the advection equation
\begin{equation}
\frac{\delta \rho}{\delta t} + \textbf{U}\cdot \bigtriangledown \rho = 0 
\end{equation}
is given by taking a first-order forward difference to approximate $ \rho_t $ and a first-order backward difference for $\rho_x $ and $ \rho_y$, which results in this equation:
\begin{enumerate}
\item Start with the advection equation:
$$
\frac{\delta \rho}{\delta t} + \textbf{U}\cdot \bigtriangledown \rho = 0 $$
and perform the del operation:
$$
\frac{\delta \rho}{\delta t} + \textbf{U}(\frac{\delta \rho}{\delta x}\vec{i}+\frac{\delta \rho}{\delta y}\vec{j}) = 0
$$
\item Then distribute \textbf{U} and multiply it times the unit directions to obtain the equation:
$$
\frac{\delta \rho}{\delta t} + u\frac{\delta \rho}{\delta x}+v\frac{\delta \rho}{\delta y} = 0
$$
\item Take the first-order forward difference to approximate $ \rho_t $
$$
\frac{\rho_{i,j}^{n+1}-\rho_{i,j}^{n}}{\Delta t} + u\frac{\delta \rho}{\delta x}+v\frac{\delta \rho}{\delta y} = 0
$$
\item Take the first-order backwards difference to approximate $ \rho_x $ and $ \rho_y $
$$
\frac{\rho_{i,j}^{n+1}-\rho_{i,j}^{n}}{\Delta t} + u\frac{\rho_{i,j}^{n}-\rho_{i-1,j}^{n}}{\Delta x}+v\frac{\rho_{i,j}^{n}-\rho_{i,j-1}^{n}}{\Delta y} = 0
$$
\item Rearrange to obtain a first-order upwind scheme:
\begin{equation}
\frac{\rho_{i,j}^{n+1}-\rho_{i,j}^{n}}{\Delta t}= - u\frac{\rho_{i,j}^{n}-\rho_{i-1,j}^{n}}{\Delta x}-v\frac{\rho_{i,j}^{n}-\rho_{i,j-1}^{n}}{\Delta y}
\end{equation}

where $ \Delta t $ is the time step and $ \Delta x $ and $ \Delta y $ are the spacial resolutions. The superscripts, n, references the time frame and the subscripts, i and j reference the spacial frame. So, n+1, denotes the subsequent time interval and n denotes the current one. i,j represents the grid node we are looking at, i+1,j is the node to the right, i-1,j is the node to the left, i,j+1 is the grid node above, and i,j-1 is the grid node below. $\rho$ represents the equation we are trying to solve for.
\end{enumerate}

Equation (2) can be rearranged to produce an iteratable equation for solving for $ \rho_{i,j}^{n+1}$:
\begin{equation}
\rho_{i,j}^{n+1} = \rho_{i,j}^{n} - \frac{u\Delta t}{\Delta x}(\rho_{i,j}^{n}-\rho_{i-1,j}^{n})-\frac{v\Delta t}{\Delta y}(\rho_{i,j}^{n}-\rho_{i,j-1}^{n})
\end{equation}

Equation (3) assumes however that the velocity vector field, \textbf{U}, is positive in both the x and y directions. Because our field \textbf{U} is not always positive in both directions, we have to modify our first-order backwards differences with respect to $u$ and $v$. So, depending on the signs of $u$ and $v$ we are left with four different schemes:
\begin {enumerate}
\item If $u$ is positive and $v$ is positive at grid node (i,j):
\begin{equation}
\rho_{i,j}^{n+1} = \rho_{i,j}^{n} - \frac{u\Delta t}{\Delta x}(\rho_{i,j}^{n}-\rho_{i-1,j}^{n})-\frac{v\Delta t}{\Delta y}(\rho_{i,j}^{n}-\rho_{i,j-1}^{n}) \tag{3.1}
\end{equation}
\item If $u$ is positive and $v$ is negative at grid node (i,j):
\begin{equation}
\rho_{i,j}^{n+1} = \rho_{i,j}^{n} - \frac{u\Delta t}{\Delta x}(\rho_{i,j}^{n}-\rho_{i-1,j}^{n})-\frac{v\Delta t}{\Delta y}(\rho_{i,j+1}^{n}-\rho_{i,j}^{n}) \tag{3.2}
\end{equation}
\item If $u$ is negative and $v$ is positive at grid node (i,j):
\begin{equation}
\rho_{i,j}^{n+1} = \rho_{i,j}^{n} - \frac{u\Delta t}{\Delta x}(\rho_{i+1,j}^{n}-\rho_{i,j}^{n})-\frac{v\Delta t}{\Delta y}(\rho_{i,j}^{n}-\rho_{i,j-1}^{n}) \tag{3.3}
\end{equation}
\item If $u$ is negative and $v$ is negative at grid node (i,j):
\begin{equation}
\rho_{i,j}^{n+1} = \rho_{i,j}^{n} - \frac{u\Delta t}{\Delta x}(\rho_{i+1,j}^{n}-\rho_{i,j}^{n})-\frac{v\Delta t}{\Delta y}(\rho_{i,j+1}^{n}-\rho_{i,j}^{n}) \tag{3.4}
\end{equation}
\end{enumerate}

\subsection{My Code}
My code uses the math presented above. Specifically, my code follows these steps:
\begin{enumerate}
\item Set number of nodes, N, equal to 100
\item Establish the domain through the variables xmin, xmax, ymin, and ymax
\item Discretise the x and y dimensions and establish variables dx and dy which correspond to $\Delta x$ and $\Delta y$ 
\item Establish a time step and simulation length.
\item Set up initial conditions and draw the first frame
\item Perform this loop until $t_{final}$ is reached.
	\begin{itemize}
	\item Calculate $u$ and $v$ for the current time $t$
	\item Calculate for the boundary conditions
	\item Update $\rho$ for $t$ + time step using Equations (3.1)-(3.4) depending on the value of $u$ and $v$
	\item Draw the consecutive frame
	\item Note: Inside the loop the steps presented above are done for each grid node individually since the velocity vector \textbf{U} has a different value at each grid node and there are conditions for updating $\rho$ that are based on \textbf{U} for a specific grid point. 
	\end{itemize}
\end{enumerate}

\section{Results}\label{sec::results}
In \texttt{MatLab}, my simulation code is invoked at the prompt by calling the script 

\texttt{AdvectionSimulation}

\texttt{AdvectionSimulation}
will run a simulation with 100 grid nodes along the x-dimension and y-dimension and 500 time steps. The value of $\rho$ is represented on the plot by the z-dimension.

See Figures 1-6 for a look at the velocity vector field \textbf{U} over time:

See Figures 7-16 for a look at the value of $\rho$ over time:



\section{Conclusion}\label{sec::conclusion}
I was not able to implement an exact solution, because I was not able to solve for $\rho$ but I believe my simulation is accurate. It is most likely first order accurate, since I implemented a first order upwind scheme as described in the \emph{Algorithms} section. Since there is no source, the value of $\rho$ decreases as it advects over time. The fluid (I take $\rho$ to symbolize a fluid) moves with the vector field and although it dissipates slowly, it never simply dissolves. Because the fluid moves as a whole in accordance with the velocity vector field over time and decreases in value slowly since there is no source, and this is the expected behavior, I believe I have implemented the simulation correctly.

\clearpage
\begin{figure}[!tbp]
  \centering
  \begin{minipage}[b]{0.49\textwidth}
    \includegraphics[width=\textwidth]{VectorField0}
    \caption{$t=0$ (initial condition)}
  \end{minipage}
  \hfill
  \begin{minipage}[b]{0.49\textwidth}
    \includegraphics[width=\textwidth]{VectorField1}
    \caption{$t=\frac{t_{final}}{4}$}
  \end{minipage}
\end{figure}
\begin{figure}[!tbp]
  \centering
  \begin{minipage}[b]{0.49\textwidth}
    \includegraphics[width=\textwidth]{VectorField3}
    \caption{$t\approx\frac{t_{final}}{2}$ (just before halfway)}
  \end{minipage}
  \hfill
  \begin{minipage}[b]{0.49\textwidth}
    \includegraphics[width=\textwidth]{VectorField4}
    \caption{$t\approx\frac{t_{final}}{2}$ (just after halfway)}
  \end{minipage}
\end{figure}
\begin{figure}[!tbp]
  \centering
  \begin{minipage}[b]{0.49\textwidth}
    \includegraphics[width=\textwidth]{VectorField5}
    \caption{$t=3\frac{t_{final}}{4}$}
  \end{minipage}
  \hfill
  \begin{minipage}[b]{0.49\textwidth}
    \includegraphics[width=\textwidth]{VectorField6}
    \caption{$t=t_{final}$}
  \end{minipage}
\end{figure}
\begin{figure}[!tbp]
  \centering
  \begin{minipage}[b]{0.49\textwidth}
    \includegraphics[width=\textwidth]{Advection10.jpg}
    \caption{$t=0$ (initial condition)}
  \end{minipage}
  \hfill
  \begin{minipage}[b]{0.49\textwidth}
    \includegraphics[width=\textwidth]{Advection11.jpg}
    \caption{$t=0$ (initial condition) aerial}
  \end{minipage}
\end{figure}
\begin{figure}[!tbp]
  \centering
  \begin{minipage}[b]{0.49\textwidth}
    \includegraphics[width=\textwidth]{Advection20.jpg}
    \caption{$t=\frac{t_{final}}{4}$}
  \end{minipage}
  \hfill
  \begin{minipage}[b]{0.49\textwidth}
    \includegraphics[width=\textwidth]{Advection21.jpg}
    \caption{$t=\frac{t_{final}}{4}$) aerial}
  \end{minipage}
\end{figure}
\begin{figure}[!tbp]
  \centering
  \begin{minipage}[b]{0.49\textwidth}
    \includegraphics[width=\textwidth]{Advection30.jpg}
    \caption{$t=\frac{t_{final}}{2}$}
  \end{minipage}
  \hfill
  \begin{minipage}[b]{0.49\textwidth}
    \includegraphics[width=\textwidth]{Advection31.jpg}
    \caption{$t=\frac{t_{final}}{2}$ aerial}
  \end{minipage}
\end{figure}
\begin{figure}[!tbp]
  \centering
  \begin{minipage}[b]{0.49\textwidth}
    \includegraphics[width=\textwidth]{Advection40.jpg}
    \caption{$t=3\frac{t_{final}}{4}$}
  \end{minipage}
  \hfill
  \begin{minipage}[b]{0.49\textwidth}
    \includegraphics[width=\textwidth]{Advection41.jpg}
    \caption{$t=3\frac{t_{final}}{4}$ aerial}
  \end{minipage}
\end{figure}
\begin{figure}[!tbp]
  \centering
  \begin{minipage}[b]{0.49\textwidth}
    \includegraphics[width=\textwidth]{Advection50.jpg}
    \caption{$t=t_{final}$}
  \end{minipage}
  \hfill
  \begin{minipage}[b]{0.49\textwidth}
    \includegraphics[width=\textwidth]{Advection51.jpg}
    \caption{$t=t_{final}$ aerial}
  \end{minipage}
\end{figure}

\appendix
\section{Implementation in MatLab}\label{sec::appendix}

The \texttt{MatLab} implementation:\newline\newline
\texttt{AdvectionSimulation.m}
\begin{verbatimtab}
close all;
clear all;

N = 100; % in each spacial direction
u(N+1,N+1) = 0;
v(N+1,N+1) = 0;
p(N+1,N+1) = 0;
p1(N+1,N+1) = 0;

%Bounds
xmin = -pi / 2;
xmax = pi / 2;
ymin = -pi / 2;
ymax = pi / 2;

%discretise x and y
dx = (xmax - xmin)/N;
x = xmin : dx : xmax;
dy = (ymax - ymin)/N;
y = xmin : dx : xmax;

%Time
timeStep = .2 * dx;
simLength = pi;
t = 0;

%%%%%%%%%%%%%%%%%% Initial conditions and initial frame %%%%%%%%%%%%%%%%%%%%%%%%
for ix = 1:N+1
    for iy = 1:N+1
        if (sqrt((x(ix)-1)^2+(y(iy)-0)^2) <= .25)
            p(iy,ix) = 1;
        end
    end
end

for ix = 1:N+1
    for iy = 1:N+1
        u(iy,ix) = -cos(x(ix))*sin(y(iy))*cos(t);
    end
end
for ix = 1:N+1
    for iy = 1:N+1
        v(iy,ix) = sin(x(ix))*cos(y(iy))*cos(t);
    end
end

mesh(x,y,p); hold on;
%quiver(x,y,u,v);
axis([-pi/2 pi/2 -pi/2 pi/2]);
xlabel('X-Dimension');
ylabel('Y-Dimension');
zlabel('Value of P');
title('Advection Simulation No Source');

%%%%%%%%%%%%%%%%%%%%%%%%%%%%%%%%%%%%%%%%%%%%%%%%%%%%%%%%%%%%%%%%%%%%%%%%%%%%%%%%

%%%%%%%%%%%%%%%%%%%%%% Loop through until simLength %%%%%%%%%%%%%%%%%%%%%%%%%%%%
for t = timeStep: timeStep: simLength
    
    % Establish u and v for time t %%%%%%%%%%%%%%%%%%%%%%%%%%%%%%%%%%%%%%%%%%%
    for ix = 1:N+1
        for iy = 1:N+1
            u(iy,ix) = -cos(x(ix))*sin(y(iy))*cos(t);
        end
    end
    for ix = 1:N+1
        for iy = 1:N+1
            v(iy,ix) = sin(x(ix))*cos(y(iy))*cos(t);
        end
    end
    U = u*timeStep/dx; %Variable to make calculations easier to type
    V = v*timeStep/dy; %Variable to make calculations easier to type
    %%%%%%%%%%%%%%%%%%%%%%%%%%%%%%%%%%%%%%%%%%%%%%%%%%%%%%%%%%%%%%%%
    
    %BOUNDARY CONDITIONS %%%%%%%%%%%%%%%%%%%%%%%%%%%%%%%%%%%%%%%%%%%%%%%%%%%%
    for iy = 1:N+1
        p(iy,1) = 0;
        p1(iy,1) = 0;
    end
    for iy = 1:N+1
        p(iy,N+1) = 0;
        p1(iy,N+1) = 0;
    end
    for ix = 1:N+1
        p(1,ix) = 0;
        p1(1,ix) = 0;
    end
    for ix = 1:N+1
        p(N+1,ix) = 0;
        p1(N+1,ix) = 0;
    end
    %%%%%%%%%%%%%%%%%%%%%%%%%%%%%%%%%%%%%%%%%%%%%%%%%%%%%%%%%%%%%

    for ix = 2:N
        for iy = 2:N
            if U(iy,ix) >= 0 && V(iy,ix) >= 0
                p1(iy,ix) = p(iy,ix) - U(iy,ix)*(p(iy,ix)-p(iy,ix-1)) ...
                           - V(iy,ix)*(p(iy,ix)-p(iy-1,ix));
            end
            if U(iy,ix) >= 0 && V(iy,ix) < 0
                p1(iy,ix) = p(iy,ix) - U(iy,ix)*(p(iy,ix)-p(iy,ix-1)) ...
                           - V(iy,ix)*(p(iy+1,ix)-p(iy,ix));
            end
            if U(iy,ix) < 0 && V(iy,ix) >= 0
                p1(iy,ix) = p(iy,ix) - U(iy,ix)*(p(iy,ix+1)-p(iy,ix)) ...
                           - V(iy,ix)*(p(iy,ix)-p(iy-1,ix));
            end
            if U(iy,ix) < 0 && V(iy,ix) < 0
                p1(iy,ix) = p(iy,ix) - U(iy,ix)*(p(iy,ix+1)-p(iy,ix)) ...
                           - V(iy,ix)*(p(iy+1,ix)-p(iy,ix));
            end
        end
    end
    
    p = p1;
    
    %%%%%%%%%%%% Plot the current state of the simulation
    clf;
    mesh(x,y,p); hold on;
%    quiver(x,y,u,v);
    axis([-pi/2 pi/2 -pi/2 pi/2 0 1]);
    xlabel('X-Dimension');
    ylabel('Y-Dimension');
    zlabel('Value of P');
    title('Advection Simulation No Source');
%    view(0,90);
    pause(.02);
    
end
%%%%%%%%%%%%%%%%%%%%%%%%%%%%%%%%%%%%%%%%%%%%%%%%%%%%%%%%%%%%%%%%%%%%%%%%%%%%%%%%

\end{verbatimtab}

\setcounter{page}{1} \pagestyle{empty}


%%%%%%%%%%%

\end{document}
